\subsection{Exercícios}

\begin{frame}{Exercícios}

	Analise as seguintes expressões lógicas e construa as tabelas verdade.

	\vspace{1cm}

	\begin{enumerate}
		\item A $\wedge$ (B $\vee$ C)
		\item A $\oplus$ B $\wedge$ C
		\item A $\vee$ B $\wedge$ C $\vee$ D
		\item (A $\vee$ B) $\wedge$ $\sim$C $\vee$ D
		% \item A $\wedge$ B v C             
		% \item A $\vee$ (B $\wedge$ $\sim$ C)         
		% \item $\sim$A $\vee$ B xor C          
		% \item A $\wedge$ B $\wedge$ $\sim$C $\vee$ D        
		% \item A $\vee$ B $\vee$ C $\vee$ $\sim$D        
		% \item $\sim$A $\wedge$ ((B $\wedge$ $\sim$C) $\oplus$ D) 
	\end{enumerate}			

	\vspace{1cm}
	\textbf{Lembrete:}
		A ordem de precedência dos operadores é:
()
$\sim$
$\wedge$
$\vee$
$\oplus$

\end{frame}


% A ^ (B v C)      
% A xor B ^ C      
% A v B ^ C v D    
% (A v B) ^ ~C v D 


% A ^ B v C
% A v (B ^ ~ C)
% ~A v B xor C
% A ^ B ^ ~C v D
% A v B v C v ~D
% ~A ^ ((B v ~C) xor D)


%
