
% \footnotesize
% \tiny


\section{Lógica Formal}
\begin{frame}{}

	\nblock{}{
		\begin{itemize}
			\item A palavra lógica vem do x que significa y.
			\item Preocupa-se, basicamente, com a estrutura do raciocínio
			% \item É o estudo dos argumentos, a ciência que separa quais argumentos são bons e quais são ruins
			\item Ela trabalha com afirmações, também chamadas de proposições, que podem ser verdadeiras ou falsas. 
			\item A um conjunto de afirmações chamamos de argumento.
			\item A análise de um argumento deve determinar se este é válido ou inválido.  
			\item Lógica é o estudo dos argumentos que busca determinar quais são válidos ou inválidos.



		\end{itemize}
	}
\end{frame}




\begin{frame}{As Leis da Lógica}
Há diversas leis que fundamentam a lógica. As mais importantes para o estudo do desenvolvimento de software são:
	\nblock{}{
		\begin{itemize}
			\item Não-contradição;
			\item Terceiro excluído;
		\end{itemize}
	}

\end{frame}



\begin{frame}{Lei do terceiro excluído}
	\nblock{}{
		Determina que uma afirmação somente pode ser verificada como verdadeira ou falsa, sem uma terceira alternativa.
	}


\end{frame}


\begin{frame}{Lei da não-contradição}
	\nblock{}{
		Determina que uma afirmação não pode ser sua negação.

		Em outras palavras,

		Uma afirmação nunca será verdadeira e falsa ao mesmo tempo.
	}

\end{frame}



\begin{frame}{Exemplos}

	\eblock{}{
		\begin{itemize}
			\item Malcon está vivo.
			\item 4 > 5.
			\item hoje é sexta-feira.
			\item A = B.
			\item Toda azeitona tem caroço.
		\end{itemize}

		\begin{itemize}
			\item As afirmações acima somente podem ser verdadeiras ou falsas.
			\item Além disso, não há outras alternativas.
		\end{itemize}
	}

\end{frame}

\begin{frame}{Axiomas}
	São afirmações autoevidentes, ou seja, não precisam de nada que as justifique.
	% que não precisam de nenhuma outra afirmação para justificá-las.
	
	\eblock{}{
		\begin{itemize}
			\item A parte é menor que o todo.
			\item Uma coisa é igual a ela mesma. (A=A)
			\item Se duas coisas forem iguais a uma terceira, então são iguais entre si. (A=C, B=C => A=B)
		\end{itemize}
	}
\end{frame}



\begin{frame}{Paradoxos}
Um paradoxo é uma afirmação que se contradiz ou que leva a uma contradição.
	\eblock{}{
		\begin{itemize}
			\item Essa afirmação é falsa.
			\item Toda regra tem exceção.
			\item Não existe verdade absoluta.
			\item Pinóquio diz: "Meu nariz vai crescer".
			\item Tudo muda.
		\end{itemize}
	}
\end{frame}


\begin{frame}{Silogismo}
	
Dados duas proposições aqui chamadas de premissas, podemos obter uma conclusão. A isso chamamos de silogismo ou lógica silogística.
	\eblock{}{
		\begin{itemize}
			\item Todo homem é mortal 
			\item Sócrates é um homem 
			\item Logo, Sócrates é mortal	
		\end{itemize}
	}

\end{frame}



% Você está numa cela onde existem duas portas, cada uma vigiada por um guarda. Existe uma porta que dá para a liberdade, e outra para a morte. Você está livre para escolher a porta que quiser e por ela sair. Poderá fazer apenas uma pergunta a um dos dois guardas que vigiam as portas. Um dos guardas sempre fala a verdade, e o outro sempre mente e você não sabe quem é o mentiroso e quem fala a verdade.
