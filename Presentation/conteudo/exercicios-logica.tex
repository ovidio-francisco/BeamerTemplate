\subsection{Exercícios}


\begin{frame}{Exercícios}

	Analise os seguintes argumentos, e verifique se são válidos:	

	\nblock{}{
		\begin{enumerate}
			\item Sempre que chove o chão fica molhado.
			\item Hoje o chão estava molhado.
			\item Logo, hoje choveu.
		\end{enumerate}			
	}


	\nblock{}{
		\begin{enumerate}
			\item Todo guitarrista é músico.
			\item Brian May é guitarrista.
			\item Logo, Brian May é músico.
		\end{enumerate}			
	}


\end{frame}



\begin{frame}{Exercícios}

	Analise os seguintes argumentos, e verifique se são válidos:	

	\nblock{}{
		\begin{enumerate}
			\item Todo programador gosta de café.
			\item Historiadores gostam de café.
			\item Logo, historiadores são programadores.
		\end{enumerate}			
	}


	\nblock{}{
		\begin{enumerate}
			\item Nicola tem nota suficiente para ser aprovado.
			\item Marie Curie tem a mesma nota que Nicola.
			\item Carl Sagan tem a mesma nota que Marie Curie
			\item Logo, Carl Sagan tem nota suficiente para ser aprovado.
		\end{enumerate}			
	}


\end{frame}



\begin{frame}{Exercícios}\footnotesize

	Analise os seguintes argumentos, e verifique se são válidos:	

	\nblock{}{
		\begin{enumerate}
			\item Todo queijo suíço tem furos.
			\item Onde tem furo não tem queijo.
			\item Quanto mais queijo tem  mais furos tem.
			\item Logo, quanto mais queijo, menos queijo.
		\end{enumerate}			
	}


	\nblock{}{
		\begin{enumerate}
			\item Deus é amor.
			\item O amor é cego.
			\item Stevie Wonder é cego.
			\item Logo, Stevie Wonder é Deus.
		\end{enumerate}			
	}

	\nblock{}{
		\begin{enumerate}
			\item Disseram que eu sou ninguém.
			\item Ninguém é perfeito.
			\item Logo, eu sou perfeito.
		\end{enumerate}			
	}


\end{frame}

\begin{frame}{Desafio}
Você está numa cela onde existem duas portas das quais pode sair livremente. Porém, uma dá para a liberdade, e outra para a morte. 

Sabendo que: 
\vspace{.5cm}

	\begin{itemize}
		\item Cada porta é vigiada por um guarda sendo que um deles \textbf{sempre fala a verdade}, e o outro \textbf{sempre mente}. 
		\item Não há como saber quem é o mentiroso e quem é o sincero. 
		\item Você pode fazer \textbf{uma }única \textbf{pergunta }a \textbf{um }dos \textbf{guardas}. 
	\end{itemize}

\vspace{.5cm}
Para sair vivo, que pergunta você faria?


\end{frame}

% Resposta: Pergunte a qualquer um deles: Qual a porta que o seu companheiro apontaria como sendo a porta da liberdade?

% Explicação: O mentiroso apontaria a porta da morte como sendo a porta que o seu companheiro e o sincero, sabendo que seu companheiro sempre mente, diria que ele apontaria a porta da morte como sendo a porta da liberdade. Conclusão: Os dois apontariam a porta da morte. Portanto, é só seguir pela outra porta












